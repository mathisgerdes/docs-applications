\documentclass[11pt]{article}

% Other packages for formatting
\usepackage[margin=1in]{geometry}
\usepackage{setspace}
\onehalfspacing
\usepackage{fancyhdr}
\usepackage{graphicx}             % For including images
\usepackage{titlesec}             % For customizing section titles

\usepackage[numbers]{natbib}


% Set up the custom page style for the first page
\fancypagestyle{firstpagestyle}{
    \fancyhf{} % Clear all headers and footers
    \fancyhead[L]{\Large \textbf{Research Statement}}
    \fancyhead[R]{\large \href{https://www.mathisgerdes.com}{\textbf{Mathis Gerdes}}}
    \renewcommand{\headrulewidth}{0pt} % Remove the default header rule
    \fancyfoot[C]{- {\thepage} -}
}

% Regular page style for the rest of the document
\pagestyle{fancy}
\fancyhf{} % Clear all headers and footers
\fancyhead[L]{\textbf{Research Statement}} % Regular header on subsequent pages
\fancyhead[R]{\textbf{Mathis Gerdes}}
\fancyfoot[C]{- {\thepage} -}

\usepackage[colorlinks=true, linkcolor=royalblue, urlcolor=royalblue, citecolor=royalblue]{hyperref}
\usepackage{xcolor}
\definecolor{royalblue}{RGB}{65, 105, 225} % Adjust the RGB values for your preferred shade of blue
\usepackage[colorlinks=true, linkcolor=royalblue, urlcolor=royalblue, citecolor=royalblue]{hyperref}

\usepackage{parskip}
\usepackage[numbers]{natbib}

% Title information
\title{}
\author{}
\date{}

\begin{document}
\thispagestyle{firstpagestyle}

As machine learning continues to drive remarkable advances across science and society, computational methods are playing an increasingly central role.
My research focuses on applying and advancing these techniques at the intersection of physics and mathematics to tackle fundamental challenges in theoretical physics.

\textbf{{Past and Ongoing Work:}}
I have leveraged my strong background in theoretical physics and computer science to build expertise across a diverse range of research projects at the intersection of machine learning and theoretical physics.
Since my MSc in Munich and throughout my PhD at the University of Amsterdam, I have led pioneering efforts in {lattice quantum field theory}, {Calabi-Yau metrics}, and {simulation-based inference}.

\textbf{\textit{{Lattice Quantum Field Theory}}}\\
Traditional sampling methods in Lattice Quantum Field Theory become inefficient near critical points. Machine learning (ML) techniques offer a potentially powerful alternative by learning complex distributions, allowing for faster and more scalable sampling.
I have developed {continuous normalizing flows} for scalar quantum field theories, expressing an ordinary differential equation as a neural network that preserves the physical symmetries of the theory \cite{gerdes2023LearningLattice}.
This advancement allowed us to scale our method to large lattice sizes while obtaining state-of-the-art sampling quality, surpassing previously published results. Our contribution has been used for example to numerically study the Nambu-Goto string.

I have recently extended this work to gauge theories, developing flexible gauge-equivariant continuous flows on Lie group-valued degrees of freedom. To achieve this, I implemented an integration scheme for matrix group manifolds that facilitates efficient gradient computation, and designed a gauge-equivariant ODE architecture that delivers state-of-the-art sampling quality. Ongoing collaboration focuses on multi-scale generation, transfer learning, and exploring the thermodynamic properties of these flows, as well as their deeper connection to the renormalization group (RG).

\textbf{\textit{{Calabi-Yau Metrics}}}\\
During my MSc and subsequent research, I developed novel machine learning methods to approximate Ricci-flat metrics on Calabi-Yau manifolds.
By leveraging an algebraic ansatz that ensures the Kähler condition, my approach provided more efficient and accurate approximations than Donaldson's algorithm, while also capturing moduli dependence \cite{gerdes2023CYJAXPackage}.
Our published summary of machine learning techniques, co-authored with Lara Anderson, James Gray, Fabian Ruehle, and my MSc supervisor Sven Krippendorf \cite{anderson2021ModulidependentCalabiYau}, has further sparked ongoing research in this field.
I have particularly enjoyed the challenge of accurately translating intricate mathematical descriptions into efficient computer implementations, as it forces a deep understanding of the former, and I hope to continue finding such challenges in my future work.

\textbf{\textit{{Physis for AI: Diffusion Models and Renormalization Group Flow}}}\\
Interest and experiments in learning real-space RG transformations have led us to investigate the parallels between RG and diffusion models. In collaboration with Max Welling, we are applying RG-inspired techniques to improve the design and understanding of diffusion-based architectures, particularly in terms of how information is erased and reconstructed at different scales. We propose a general unified framework of diffusion processes, conceptually bridging the gap between autoregressive models and diffusion.

\textbf{\textit{{Simulation-Based Inference and Stellar Streams}}}\\
Novel simulation-based inference methods using machine learning offer a powerful framework for handling detailed theoretical models that include large numbers of nuisance parameters.
In this context, I have contributed to a study of stellar streams which may yield insight into the nature and distributions of dark matter sub-halos in the Milky-Way by analyzing astrophysical data \cite{alvey2023AlbatrossScalable}.
I am currently investigating novel optimization schemes that may reconcile certain conflicts between frequentist and Bayesian approaches in this field. \textcolor{blue}{too tangential/unspecific?}


\paragraph{\color{royalblue}{Research Plans.}}
Having gained significant experience in both developing and applying machine learning tools to solve theoretical physics problems, I aim to continue pushing the boundaries of computational techniques in physics.

I plan, in particular, to continue innovating in ML techniques, such as normalizing flows and generative models, for problems in quantum field theory and lattice computations.
In this I hope to build both on my existing network of collaborators, and to make new connections within the IAIFI network.
More abstractly, I will continue exploring ideas connecting theoretical physics, and RG flow in particular, with generative models.
% In these endeavors, I aim not only to solve existing challenges but also to pioneer new methods that will significantly enhance computational physics as a whole.

\bibliographystyle{plainnat}
\bibliography{refs}

\end{document}
