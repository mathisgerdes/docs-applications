\documentclass[11pt]{article}

% Other packages for formatting
\usepackage[margin=1in]{geometry}
\usepackage{setspace}
\linespread{1.02}
% \onehalfspacing
\usepackage{fancyhdr}
\usepackage{graphicx}             % For including images
\usepackage{titlesec}             % For customizing section titles

\usepackage{amsmath, physics, amssymb}
\usepackage{parskip}

% Set up the custom page style for the first page
\fancypagestyle{firstpagestyle}{
    \fancyhf{} % Clear all headers and footers
    \fancyhead[L]{\LARGE \textbf{Cover Letter}}
    \fancyhead[R]{\Large \href{https://www.mathisgerdes.com}{\textbf{Mathis Gerdes}}}
    \renewcommand{\headrulewidth}{0pt} % Remove the default header rule
    % \fancyfoot[C]{- {\thepage} -}
}

% Regular page style for the rest of the document
\pagestyle{fancy}
\fancyhf{} % Clear all headers and footers
\fancyhead[L]{\textbf{Cover Letter}} % Regular header on subsequent pages
\fancyhead[R]{\textbf{Mathis Gerdes}}
% \fancyfoot[C]{- {\thepage} -}

\usepackage{xcolor}
\definecolor{royalblue}{rgb}{0.2, 0.3, 0.7} % Adjust the RGB values for your preferred shade of blue
\usepackage[colorlinks=true, linkcolor=royalblue, urlcolor=royalblue, citecolor=royalblue]{hyperref}

% Title information
\title{}
\author{}
\date{}

\begin{document}
\thispagestyle{firstpagestyle}

% Header with right-aligned personal information
\noindent
\begin{minipage}[t]{0.5\textwidth}
\phantom{
\today \\
}
\vspace{0.2cm} \\
Center for Theoretical Physics \\
MIT \\
77 Mass Ave, 6-307 \\
Cambridge, MA 02139

\end{minipage}
\begin{minipage}[t]{0.5\textwidth}
\flushright
\today \\
\vspace{0.2cm}
University of Amsterdam \\
Science Park 904 \\
Amsterdam 1098 XH \\
The Netherlands \\
\end{minipage}

\vspace{20pt}

\noindent
\textbf{Dear Members of the MIT Center for Theoretical Physics Selection Committee,}

I am writing to express my strong interest in the postdoctoral position in high-energy theory at the MIT Center for Theoretical Physics (CTP). Currently, I am completing my PhD in Theoretical Physics at the University of Amsterdam under the supervision of Miranda C. N. Cheng and Christoph Weniger. My research lies at the intersection of quantum field theory, string theory, and machine learning, with a focus on lattice quantum field theory and advancing machine learning approaches in theoretical physics.

At the CTP, I am particularly excited about the opportunity to collaborate with members such as Phiala Shanahan and Tracy Slatyer, whose work on lattice QCD and simulation-based inference respectively overlap closely with my own work in developing novel computational techniques for theoretical physics. My research aligns closely with the group's broad interests, and I bring a strong interdisciplinary background to contribute to the CTP's dynamic intellectual environment.

In my work on lattice quantum field theory, I developed a class of equivariant continuous normalizing flows with the goal of efficient sampling from high-dimensional distributions by making use of symmetries inherent to the target theories. These methods have demonstrated state-of-the-art performance in scalar and pure gauge lattice field theories in two dimensions.

In string theory, I have pioneered the application of machine learning for approximating Ricci-flat metrics on Calabi-Yau manifolds. By incorporating moduli dependence into a spectral ansatz parametrized as a neural network, I have achieved higher accuracy and computational efficiency compared to traditional numerical approaches.

While my primary research lies in high energy physics, I am also keen to explore adjacent areas where advanced computational methods can yield new insights. For example, I have implemented a simulation-based inference pipeline to analyze stellar streams, which probe dark matter structure in the Milky Way. This work aligns in particular with Tracy Slatyer's expertise in using astrophysical observations to inform fundamental physics.

As a postdoctoral researcher at the CTP, I aim to advance my work on non-perturbative QFT and machine learning methods, while exploring collaborative opportunities across the center's vibrant research community. I am particularly excited about contributing to foundational problems in high-energy theory and bridging computational tools with theoretical challenges.

Thank you for considering my application. I look forward to the opportunity to contribute to the CTP's dynamic research environment and discuss how my expertise aligns with its vision.


% \vspace{5pt}
\noindent
\flushright
Sincerely, \\
Mathis Gerdes \\
\href{mailto:m.gerdes@uva.nl}{m.gerdes@uva.nl}

\end{document}
